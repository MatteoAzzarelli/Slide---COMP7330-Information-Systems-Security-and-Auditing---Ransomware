\documentclass{beamer}
\usepackage[english,italian]{babel}

\usepackage[T1]{fontenc}
\usepackage{textcomp}
\usepackage[utf8]{inputenc}	%caratteri accentati

\newcommand\tab[1][1cm]{\hspace*{#1}}	% define new command tab

\usepackage[font=scriptsize]{subcaption}
\usepackage[most]{tcolorbox}

%insert code
\usepackage{listings}

%------------------------------------------------
%		Trasparenza liste
%------------------------------------------------
\setbeamercovered{transparent}


%------------------------------------------------
%		IMAGE DEFINITIONS
%------------------------------------------------
\usepackage{changepage}

\usepackage{pgf,tikz}
\usepackage{graphicx}

 \pgfdeclareimage[height = 7cm]{img:wannacry}{./images/wannacry_01.png}
% \pgfdeclareimage[height=4cm]{socialNetwork}{./images/Graph/socialNetwork}


\definecolor{blueHKBU}{RGB}{0,0,210}

\usetheme{Padova}

\title{\vspace{1cm}Ransomware\\\Large [Malicious Software]}
\subtitle{\vspace{-0cm}{\large COMP7330 Information Systems Security and Auditing}\\ \vspace{0.5cm} \large GROUP 11}
\author{%\vspace{0.5cm}%\textbf{Laureando}:\\
	Matteo Azzarelli \hspace{5cm} Nurzhan Izdauov}
% 	Matteo Azzarelli\\ Nurzhan Izdauov}
\date{\vspace{2.7cm}\color{black}{\footnotesize{10 April 2019}}}




\begin{document}

    \AtBeginSection[]{
      \begin{frame}{~}
      \vfill
      \centering
      \begin{tcolorbox}[halign=center,colframe=blueHKBU, colback=blueHKBU!5!white]
        \usebeamerfont{title}\insertsectionhead\par%
      \end{tcolorbox}
      \vfill
      \end{frame}
    }
    
	\maketitle

% 	\begin{frame}{Outline}
% 		\tableofcontents
% 	\end{frame}

 	\section{Introduction}
	
	\begin{frame}{Introduction}
		Ransomware is a type of malicious softWARE when the victim is forced to pay some RANSOM to solve the consequences:

		\begin{itemize}
			\item<2-> Locker Ransomware - encrypts the whole drive and essentially locking the user out of the system.
			\item<3-> Crypto Ransomware - encrypts and locks only specific data which seems to be important.
		\end{itemize}
	\end{frame}
	
	\begin{frame}{Ex. WannaCry}
		\vspace{-0.5cm}
		\begin{center}
			    \pgfuseimage{img:wannacry}
		    \end{center}
	\end{frame}
	
	\begin{frame}{Four signs}
	    Four signs:
		\begin{enumerate}[<+->]
			\item Splash screen blocks access - the most obvious sign.
			\item Files that will not open - might be a sign of encryption attack on your file.
            \item Odd or missing file extensions - this files may be masked ransomware.
            \item Received instructions for paying the ransom - the second most obvious sign.
		\end{enumerate}
	\end{frame}
	
	\begin{frame}{Guidelines}
	    Guidelines to prevent you from this attack:
		\begin{itemize}[<+->]
			\item Never download files from unknown sources.
            \item Backup your data on a regular basis.
            \item Install the latest anti-viruses and update them periodically.
            \item Never pay the ransom - there is no guarantee you will get your files back.
            \item<4-> Etc.
		\end{itemize}
	\end{frame}
	
	\section{Possible Damages}
	
	\begin{frame}{Possible Damages}
	    Ransomware can affect \textbf{home users} but also \textbf{Businesses}.\\
	    \pause
	    Possible damages:
		\begin{enumerate}[<+->]
		    \item Temporary or permanent loss of sensitive or proprietary information;
		    \item Lost profits caused by downtime;
			\item Cost of replacing compromised devices;
			\item Reputational damage;
			\item Recovery costs;
			\item Potential legal penalties;
			\item Employee training in response to attacks;
		\end{enumerate}
	\end{frame}
	
	\begin{frame}{References}
		\begin{itemize}
		    \item https://www.police.gov.hk/ppp\_en/04\_crime\_matters/tcd/\\types\_11.html
			\item https://www.metacompliance.com/blog/dangers-of-ransomware/
			\item https://security.berkeley.edu/faq/ransomware/what-possible-impact-ransomware
			\item https://us.norton.com/internetsecurity-malware-ransomware.html 
            \item https://www.computerworld.com/article/3105344/4-signs-youre-a-victim-of-ransomware.html 

		\end{itemize}
	\end{frame}
	
% 	\begin{frame}{More models}
% 	    There are many variants of this model like:
%         \begin{itemize}[<+->]
%             \item \textbf{SEIRS}\only<1>{: this model provide the Exposed(E) class where the person has the disease but it doesn’t spread it on to susceptible nodes.} \only<1,4>{This kind of model is suitable for disease with a \textbf{long incubation disease}.}\only<1>{
%             \begin{center}
% 			    \pgfuseimage{seirModel}
% 		    \end{center}}
%             \item \textbf{SIS}\only<2>{: here there are only three states and modelling disease that after the state of infection the people come back to the state of susceptible.
%             \begin{center}
% 			    \pgfuseimage{sisModel}
% 		    \end{center}
%             }
%             \item \textbf{SIRS}\only<3>{: this one is a variant of the previous model that introduce a stage of immunity for a certain time after that the person can be infected again.
%             \begin{center}
% 			    \pgfuseimage{sirsModel}
% 		    \end{center}
%             }
%         \end{itemize}

%         \only<4>{The previous last two models are typically used to represent different \textbf{epidemics disease} such as HIV or SARS [3].}
% 	\end{frame}
	


	\begin{frame}{Thanks}
		\begin{center}
			{\Huge \textcolor{rossoPantano}{\textbf{THANKS}}}
		\end{center}
		
	\end{frame}

\end{document}
